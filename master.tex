%This document wants to the explain the quran package with some examples.
\documentclass{report}
\usepackage{hyperref}
\hypersetup{
    bookmarks=true,         % show bookmarks bar?
    unicode=true,          % non-Latin characters in Acrobat’s bookmarks
    pdftoolbar=true,        % show Acrobat’s toolbar?
    pdfmenubar=true,        % show Acrobat’s menu?
    pdffitwindow=false,     % window fit to page when opened
    pdfstartview={FitH},    % fits the width of the page to the window
    pdftitle={Quran},    % title
    pdfauthor={Author},     % author
    pdfsubject={Subject},   % subject of the document
    pdfcreator={Creator},   % creator of the document
    pdfproducer={Producer}, % producer of the document
    pdfkeywords={keyword1, key2, key3}, % list of keywords
    pdfnewwindow=true,      % links in new PDF window
    colorlinks=true,       % false: boxed links; true: colored links
    linkcolor=blue,          % color of internal links (change box color with linkbordercolor)
    citecolor=green,        % color of links to bibliography
    filecolor=magenta,      % color of file links
    urlcolor=cyan           % color of external links
}
\usepackage{forloop}
\usepackage{quran}
%The xepersian package automatically load bidi, and I've loaded it because I want to set a font that supports Arabic letters
\usepackage{xepersian}

% This macro set the main text font for non-latin letter, and it can scale font.
\settextfont[Scale=2]{Scheherazade}

\def\surna[#1]{\centerline{\hss\surahname[#1]\hss\lr{\surahname[#1]}\hss}}
\def\test#1{
    \par
    \surna[#1]
    \quransurah*[#1]
    \bigskip
}

\title{Quran Translation}
\author{Abdulla Yusufali}
\begin{document}
\maketitle
\tableofcontents
\newpage
% For typesetting بِسمِ اللَّهِ الرَّحمٰنِ الرَّحيمِ use below macro
\input{qu.tex}

%\quransurah[108] % Surah Al-Kauther

%%\surna[110]\quransurah*[110]  % Surah Al-Nasr

%The below typeset 104th surah through 113th surah.
%%\quransurah*[104-113]

%\makeatletter
%\surna[\qt@surah@default]\quransurah*  % Surah Al-Ikhlas
%\makeatother

%\quranayah[33][33]
%\quranayah*[76][1-22]

%%You can typeset whole of Holy Quran with below commands.
%%\newcounter{ct}
%%\forloop{ct}{1}{\value{ct} < 115} {\test{\value{ct}}}

%%You can typeset whole of Holy Quran with below commands.
%%\newcounter{jz}
%%\forloop{jz}{1}{\value{jz} < 31} {\quranjuz[\value{jz}]}

%%\quranjuz*[28-30]
%\hfill  صفحة  \arabic{pg} \par
%\quranpage*[\value{pg}]\vfill}

%\quranhizb*[117-120]

%\quranquarter*[1-4]
%\quranquarter*[239-240]

%\quranruku[313]
%\quranruku[556]

%\quranmanzil*[2]


%%\surna[1]\qurantext  % Surah Al-Hamd

%%\surna[1]\qurantext* % Surah Al-Hamd

%%\surna[114]\qurantext[6231-6236]  % Surah Al-Nas

%%\surna[114]\qurantext*[6231-6236]  % Surah Al-Nas

%%%\surna[2]\qurantext*[8-293] % Surah Al-Baqara

%\qurantext[1-6236] % The whole of Holy Quran
\end{document}
